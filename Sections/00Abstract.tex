

An animal's utilization distribution describes the probability of finding the animal at a given point in space. It is a tool that can be used to gain information about the habitat preferences of an animal. There exist many ways to model utilization distributions. The Langevin movement model is one model proposed for animal movement that does this. This model is based on the Langevin diffusion process. This process has an explicit stationary distribution that is the utilization distribution; thus, it can be used to model the long-term spatial distribution of an animal by looking at its displacement over time. To model data, the Langevin process is discretized using the Euler-Maruyama method. There is an error in this approximation and, for long time intervals between observations, this error leads to a bias in the utilization distribution parameter estimates. In this thesis, I study three methods for obtaining improved estimates for the Langevin movement model and find that using Monte Carlo integration over intermediate states between observations can be used to remove the bias. Using this method, the Langevin movement model can be applied to data where there are large intervals in time between observations. In addition, I show that when the sampling intervals are longer, there is a reduced variance in the utilization distribution parameter estimates. This means that the method can be used to improve animal tracking experiment designs by sampling at lower frequencies. 






\newpage
\null
\thispagestyle{empty}
\newpage

\chapter*{Sammendrag}













\newpage
\null
\thispagestyle{empty}
\newpage