

An animal's utilization distribution describes the probability of finding the animal at a given point in space. It is a tool that can be used to gain information about the habitat preferences of an animal. There exist many ways to model utilization distributions. The Langevin movement model is one model proposed for animal movement that does this. This model is based on the Langevin diffusion process. This process has an explicit stationary distribution, that is, the utilization distribution. Thus, it can be used to model the long-term spatial distribution of an animal by looking at its displacement over time. To model data, the Langevin process has to be discretized. Conventional methods for discretizing the Langevin process cause an error, and for long time intervals between observations, this error leads to a bias in the utilization distribution parameter estimates. In this thesis, I study three methods for obtaining improved estimates for the Langevin movement model and find that integrating over unobserved intermediate states can be used to remove the bias. Using this method, the Langevin movement model can be applied to data where there are large intervals in time between observations. In addition, I find that when the sampling intervals are longer, there is a reduced variance in the utilization distribution parameter estimates. This means that the method can be used to improve animal tracking experiment designs by sampling at lower frequencies. 






\newpage
\null
\thispagestyle{empty}
\newpage

\chapter*{Sammendrag}

Et dyrs utnyttelsesfordeling beskriver sannsynligheten for å finne dyret på et gitt punkt i rommet. Det er et verktøy som kan brukes for å oppnå informasjon om dyrets habitatpreferanser. Det finnes mange måter å modellere utnyttelsesfordelinger på. Langevin-bevegelsesmodellen er en modell som er foreslått for å beskrive dyrebevegelser som gjør dette. Modellen er basert på Langevin diffusjons-prosessen. Dette er en prosess som har en eksplisitt stasjonærfordeling som er utnyttelsesfordelingen. Derfor kan den brukes til modellerer den lang-tids romlige fordelingen til etter dyr ved å se på dyrets bevegelser. For å modellere data må Langevin-prosessen diskretiseres. Konvensjonelle metoder for diskretisering av Langevin-prosessen fører til en feil, og for lange tidsintervaller mellom observasjoner fører denne feilen til en bias i parameterestimatene for utnyttelsesfordelingen. I denne oppgaven studerer jeg tre metoder for å oppnå forbedrede estimater for Langevin-bevegelsesmodellen, og finner at det å integrere over uobserverte mellomliggende posisjoner kan brukes til å fjerne denne biasen. Ved å bruke denne metoden kan Langevin-bevegelsesmodellen anvendes på data der det er store tidsintervaller mellom observasjonene. I tillegg finner jeg at når intervallene mellom observasjonene er lengre, reduseres variansen i parameterestimatene for utnyttelsesfordelingen. Dette betyr at metoden kan brukes til å forbedre utformingen av dyresporingsforsøk ved å å samle inn data ved en lavere frekvens.











\newpage
\null
\thispagestyle{empty}
\newpage