In this chapter I present theory that will be used throughout the thesis.



\section{Stochastic Differential Equations}
Stochastic differential equations can be used to describe how stochastic variables change with time. These are widely used to model phenomena in areas as diverse as finance, ecology, epidemeology, neurology, geology and physics \cite{iacus_simulation_2008}. The following section introduces the concept of a stochastic diffusion equation, which is a form of Stochastic differential equation, and the preliminaries that are necessary to define such an equation.


\subsection{Brownian Motion}
Brownian motion is a continuous stochastic process can be characterized as the limit of a random walk process where the Gaussian step's length tend to 0. A stochastic process $B = \{B_t:t\geq 0\}$ is Brownian motion if it has continuous paths and independent Gaussian increments, such that $B_0 = 0$ with probability 1, $E[B_t] = 0$ and $Var(B_t - B_s) = t-s$, for $0 < s < t$(\cite{iacus_simulation_2008}).  That $B_t$ is a continuous path means that

Using this definition, we can define the k-dimensional standard Brownian motion starting at $\textbf{x} \in \mathbb{R}^k$ as the k-dimensional vector $\mathbf{B}^\mathbf{x} = (B^{x_1}, \dots , B^{x_k})$, containing standard Brownian motions $B^{x_i}$ with starting points $x_i$\cite{bhattacharya_continuous_2023}.

Lastly we can define an even more general version of the Brownian motion. That is, given a $k \times k$ matrix $\pmb{\sigma}$ and a k-dimensional vector $mu$, we can define the stochastic process $\mathbf{X}_t^\mathbf{x} = \mathbf{x} + \pmb{\mu}t + \pmb{\sigma} \mathbf{B}$ as the k-dimensional Brownian motion starting at $\mathbf{x} \in \mathbb{R}^k$, with drift coefficient $\pmb{\mu}$ and diffusion coefficient $\mathbf{D} = \pmb{\sigma} \pmb{\sigma}^T$ \cite{bhattacharya_continuous_2023}.

\subsection{Stochastic Integrals}
The integral of a function with respect to Brownian motion is known as an Itô integral. 
$$I(X) = \int_0^T X(u)dB_u$$

\cite{iacus_simulation_2008}


Let $f(t)$ be a non-stochastic continuously differentiable function. Then

$$\int_\alpha^\beta f(s)dB_s = f(\beta)B_\beta - f(\alpha)B_\alpha - \int_\alpha^\beta B_sf'(s) ds   $$



\subsection{Diffusion Processes}
\label{sec: diffusion processes}
We can use Brownian motion to define a type of SDE called a stochastic diffusion equation. This defines a stochastic diffusion process as the solution to the equation

\begin{equation}
    d{\textbf{X}}_t = {\pmb{\mu}} ({\textbf{X}}_t )dt + \pmb{\sigma}({X}_t ) d\textbf{B}_t, \   t > 0 , \ \textbf{X}(0) = \textbf{X}_0
\end{equation}

where $\textbf{B}_t$ is a k-dimensional standard Brownian motion, and $\textbf{X}_t$ is a k-dimensional stochastic variable. $\pmb{\mu}$ is a k-dimensional vector known as the drift of the equation, and $\pmb{\sigma}(t)$ is a $k\times k$-dimensional matrix, where $\pmb{\sigma}(t)\pmb{\sigma}(t)^T$ is known as the diffusion matrix. The displacements $X_{t+dt} - X_t$ are approximately gaussian when $\pmb{\mu}$ and $\pmb{\sigma}^2(x)$ are sufficiently smooth, with mean $\pmb{\mu}(\textbf{X}_t)dt$ and variance $\pmb{\sigma}(\textbf{X}_t)\pmb{\sigma}(\textbf{X}_t)^Tdt$ \cite{bhattacharya_continuous_2023}




\subsection{Euler-Maruyama  Discretization}
\label{subsec: Euler-Maruyama}
Often there is no analytical solution available for SDEs. In these cases the distribution which results from the SDE has to be approximated by using discretization schemes.  One of the most popular approximation schemes for SDEs is the Euler-Maruyama discretization, which is inspired by Euler's method for initial value problems. The approximation can be applied for any SDE of the form

\begin{equation}
    d\textbf{X}_t = \mu(t, \textbf{X}_t)dt + \sigma(t, \textbf{X}_t)d\textbf{B}_t
\end{equation}

Where  $\textbf{B}_t$ is the standard d-dimensional Brownian motion. For such equations it defines a continuous approximate solution $Y(t)$ on a discretized time interval $0=t_0 < t_1 < \dots < t_N = T$. At the discretization points the approximation is given by

\begin{equation}
    \textbf{X}_{i+1} = \textbf{X}_i + \mu(t_i, \textbf{X}_i)(t_{i+1} - t_i) + \sigma(t_i, \textbf{X}_i)(\textbf{B}_{i+1} - \textbf{B}_i)
    \label{eq: euler approximation}
\end{equation}




\section{First and Second Derivative Approximation}
\subsection{First Derivative}
\label{subsec: gradient estimation}
On multiple occasions in this thesis i use first and second derivatives to find the gradient and hessian of spatial covariates. These covariates will be represented by values on a regular grid, instead of a function. This grid does not in itself have derivatives, but the grid can be found using polynomial interpolation. For the first order derivatives i follow \parencite{michelot_langevin_2019} and use bi-linear interpolation to find an approximation. and i approximate the second derivatives using bi-cubic interpolation. 

\

Let $f:\mathbb{R}^2\rightarrow \mathbb{R}$ be the function that we want to find the derivative of at a point $(x,y) \in \mathbb{R}^2$, when we know the value of $f$ at points $(x_1, y_1), (x_2, y_1), (x_1, y_2), (x_2, y_2) \in \mathbb{R}^2$ where $x_2 < x < x_1$ and $y_2 < y < y_1$. We can write the known values of $f$ as 


$$
\begin{array}{lcl}
     f(x_1, y_1)& = & f_{11}  \\
     f(x_1, y_2)& = & f_{12}  \\
     f(x_2, y_1)& = & f_{21}  \\
     f(x_2, y_2)& = & f_{22}  
\end{array}
$$
The bilinear interpolation of $f$ in $(x,y)$ is  

$$
\hat{f}(x,y) = \frac{y_2-y}{y_2-y_1}(\frac{x_2-x}{x_2-x_1}f_{11} + \frac{x-x_1}{x_2-x_1}f_{21}) + \frac{y-y_1}{y_2-y_1}(\frac{x_2-x}{x_2-x_1}f_{12} + \frac{x-x_1}{x_2-x_1}f_{22})
$$

The partial derivatives of the  interpolated function are

$$
\begin{array}{lcl} 
    \frac{\partial\hat{f}}{\partial x} & = & \dfrac{(y_2-y)(f_{21}-f_{11}) + (y-y_1)(f_{22}- f_{12})}{(y_2-y_1)(x_2-x_1)} \\
    \frac{\partial\hat{f}}{\partial y} & = & \dfrac{(x_2-x)(f_{12}-f_{11}) + (x-x_1)(f_{22}-f_{21})}{(y_2-y_1)(x_2-x_1)}
\end{array}
$$

\subsection{Second Derivative}
\label{subsec: second derivative}
For the second derivative, assume we have $x_{-1} < x_0 < x < x_1 < x_2$ and $y_{-1} < y_0 < y < y_1 < y_2$, and define $f_{ij} = f(x_i, y_j)$, $f_{xij} = f_x(x_i,y_j)$, $f_{yij} = f_y(x_i,y_j)$ and $f_{xyij}$ for $i,j = 0, 1$. When we have the values for $f(x,y), f_x(x,y), f_y(x,y), f_{xy}(x,y)$ at the positions $(x_0, y_0), (x_1, y_0), (x_0, y_1), (x_1, y_1)$. We can find the bi-cubic interpolation polynomial by

$$
\hat{f}(x,y) = \sum_{i=0}^3 \sum_{j=0}^3 a_{ij}x^i y^j
$$

where $a_{ij}$ are the indexes of the matrix $A$, which can be found by solving

$$
A = \begin{pmatrix}
    1&0&0&0 \\
    0&0&1&0 \\
    -3&3&-2&-1 \\
    2&-2&1&1
\end{pmatrix} \begin{pmatrix}
    f_{00}&f_{01}&f_{y00}&f_{y01} \\
    f_{10}&f_{11}&f_{y10}&f_{y11} \\
    f_{x00}&f_{x01}&f_{xy00}&f_{xy01} \\
    f_{x10}&f_{x11}&f_{xy10}&f_{xy11} \\
\end{pmatrix} \begin{pmatrix}
    1&0&-3&2 \\
    0&0&3&-2 \\
    0&1&-2&1 \\
    0&0&-1&1
\end{pmatrix}
$$


we can then find the second derivatives by using


$$
\begin{array}{rcl}
    \hat{f}_{xx}(x,y) & = & \sum_{i=2}^3 \sum_{j=0}^3 i (i-1) a_{ij}x^{i-2} y^j \\
    \hat{f}_{xy}(x,y) & = & \sum_{i=2}^3 \sum_{j=0}^3 i j a_{ij}x^{i-1} y^{j-1} \\
    \hat{f}_{yy}(x,y) & = & \sum_{i=2}^3 \sum_{j=0}^3 j (j-1) a_{ij}x^{i} y^{j-2}
\end{array}
$$

The values $f_{xij}$, $f_{yij}$ and $f_{xyij}$ are not available to us, so they have to be estimated by finite difference methods. 


$$
\begin{array}{rcl}
    f_{xij} & \approx & \dfrac{f_{i+1,j} - f_{i-1,j}}{x_{i+1} - x_{i-1}}  \\
    f_{yij} & \approx & \dfrac{f_{i,j+1} - f_{i,j-1}}{y_{j+1} - y_{j-1}} \\
    f_{xyij} & \approx & \dfrac{f_{i+1,j+1} - f_{i+1,j-1} - f_{i-1,j+1} + f_{i-1,j-1}}{(x_{i+1} - x_{i-1})(y_{j+1} - y_{j-1})}
\end{array}
$$


\parencite{choudhary_bicubic_2018}

\section{Markov Chain Monte Carlo Methods}


The idea of Markov chain Monte Carlo methods is to sample from a distribution using a Markov chain which is constructed in such a way that its stationary distribution is that of the target distribution. The sample can then be used to make inferences from the target distribution. One method of constructing such a Markov chain is by using the Metropolis algorithm(CITATION). This algorithm uses a proposal kernel $p(\phi | \theta)$ to find a new value $\phi$ given the previous value $\theta$. A sufficient condition for this kernel to converge to the target distribution $f_t$ is that the chain is reversible, meaning $f_t(\theta) p(\theta|\phi) = f_t(\phi) p(\phi|\theta)$ for all $\phi,\theta$ in the state space (gamerman and lopes). The Metropolis-Hastings algorithm introduced in (CITATION) achieves this by using a proposal density $q(\phi|\theta)$ and a rejection probability $\alpha$. The proposed value is accepted as a sample-observation at a probability $1-\alpha$, which ensures that the reversibility condition is satisfied. The rejection probability that the Metropolis-Hastings algorithm proposes is

$$
    \alpha = min \left( 1, \frac{f_t(\phi)q(\phi|\theta)}{f_t(\theta)q(\theta|\phi)} \right) = min \left( 1, \frac{\pi(\phi)L(X,Y|\phi)q(\phi|\theta)}{\pi(\theta)L(X,Y|\theta)q(\theta|\phi)} \right)
$$

There are many possible options for the proposal density $q(\phi|\theta)$. For example $q$ can be independent of the previous value giving what is called an Independence proposal. This works best when the proposal density is close to the target density. Another one is having the distribution of proposed values centered around the current value, which is called a random walk proposal. In this case we can use distributions like the uniform distibution with center at the current value, or normal distribution with mean at the current value. There are also more advanced methods that have been developed like the "No-U" sampler available in the R-package "rstan".


\section{The Extended Kalman Filter}
\label{sec: EKF}

The Kalman filter gives an estimate for the states of a state-space model. A state-space model has two components: observations $Z_k$ which we observe, and states $X_k$ which are hidden. These are modeled using a transition equation and an observation equation

$$
\begin{array}{lcl}
Z_k &=& h(X_k) + v_k
\\
X_{k+1} &=& f(X_k, u_k) + w_k
\end{array}
$$

 $v_k$ and $w_k$ are multivariate Gaussian distributions with zero mean and covariance matrices $R_k$ and $Q_k$ respectively. $f$ and $h$ are functions which are referred to as the transition function and the observation function, and $u_k$ is an input into the hidden states' transition, which is independent from the hidden state. The transition from one state to another is only dependent on the previous state, so the states form a Markov chain. The observations are only dependent on the current state, so the relationship can be visualized as a graph

 \

\begin{center}
\begin{tikzpicture}[main/.style = {draw, circle}]
%Nodes
\node[main]        (state0)                              {$X_0$};
\node[main]        (state1)       [right=of state0] {$X_1$};
\node[main]        (staten)       [right=of state1] {$X_n$};
\node[main]        (obs0)       [below=of state0] {$Z_0$};
\node[main]        (obs1)       [below=of state1] {$Z_0$};
\node[main]        (obsn)       [below=of staten] {$Z_0$};
\node              (control0)       [above=of state0] {$u_0$};
\node              (control1)       [above=of state1] {$u_1$};
\node              (controln)       [above=of staten] {$u_n$};


%Lines
\draw[->] (state0.east) -- (state1.west);
\draw[->] (state0.south) -- (obs0.north);
\draw[->] (state1.south) -- (obs1.north);
\draw[->] (staten.south) -- (obsn.north);
\draw[->] (staten.south) -- (obsn.north);
\draw[->] (control0.south) -- (state0.north);
\draw[->] (control1.south) -- (state1.north);
\draw[->] (controln.south) -- (staten.north);
\draw[dashed, ->] (state1.east) -- (staten.west);
\end{tikzpicture} 
\end{center}
\ 
 
 In the case that $f$ and $h$ are linear we can use the Kalman filter to find the likelihood of the observations of the state space model. In this thesis, the transition function used is not linear. What can be used instead is the extended Kalman filter, which is a nonlinear extension of the Kalman filter. Using the extended Kalman filter we get an estimate $\hat{X}_{n|m}$ for the hidden state $X_n$ using the the observations up to $Z_m$


\

\textbf{predict:}

Predicted state estimate: $\bm{\hat X}_{k|k-1} = f(\bm{\hat X}_{k-1|k-1} )$

Predicted estimate covariance: $\textbf{P}_{k|k-1} = \textbf{F}_k\textbf{P}_{k-1|k-1} \textbf{F}_k^T + \textbf{Q}_k$

\

\textbf{update:}

Innovation: $\bm{\tilde y}_{k} = \textbf{Z}_k - h(\bm{\hat X}_{k|k-1})$

Innovation covariance: $\textbf{S}_k = \textbf{H}_k \textbf{P}_{k|k-1}\textbf{H}_k^T + \textbf{R}_k$

Optimal Kalman gain: $\textbf{K}_k = \textbf{P}_{k|k-1} \textbf{H}_k^T \textbf{S}_k^{-1}$

Updated state estimate: $\bm{\hat X}_{k|k} = \bm{\hat X}_{k|k-1} + \textbf{K}_k \bm{\tilde y}_{k}$

Updated estimate covariance: $\textbf{P}_{k|k} = (\textbf{I} - \textbf{K}_k \textbf{H}_k)\textbf{P}_{k|k-1}$



\

where  $F_k = Jac (f)(\hat{X}_{k-1|k-1})$ and $H_k = Jac( h)(\hat{X}_{k-1|k-1})$

\

using the state and covariance estimates we get the likelihood 

$$L = \prod_{k=1}^K \mathcal{N}(Z_k; h(\hat{x}_{k|k-1}), S_k) \label{eq: EKF likelihood}$$

Where $K$ is the number of states.
\parencite{kulikov_extended_2024}

\begin{comment}

    \section{Poisson Processes}
A Poisson process is a stochastic process which consists of individual points scattered in a space. This scattering is uniform, with a rate at which the points occur. The poisson process is used extensively in statitsics to describe a broad range of natural phenomena, for example the occurrence of animal observations. \parencite{Pinsky2011} defines the Poisson process by 

\

Definition: Let $S \subset \mathbb{R}^k$, $\mathscr{A}$ be the family of subsets of $S$ and for any set $A\in \mathscr{A}$, let $|A|$ denote the size of A. Then ${N(A):A\in \mathscr{A}}$ is a homogeneous Poisson process of intensity $\lambda$ if (1) for each $A \in \mathscr{A}$, the random variable $N(A)$ has a Poisson distribution with parameter $\lambda|A|$, and (2) for each finite collection $\{ A_1, \dots A_n\}$ of disjoint subsets of $S$, the random variables $N(A_1), \dots, N(A_n)$ are independent. 

\
The poisson distribution is defines as 


$$
P(N=n) = \frac{e^{-\mu}\mu^n}{n!}
$$


The homogeneous Poisson process can be extended to what is called the inhomogeneous Poisson point process. instead of having a constant intensity $\lambda$, we have an intensity function $\lambda(s)$ which varies with space or time. 

An inhomogeneous Poisson point process is characterized by its intensity function $\lambda(s)$, and that the number of points in disjoints sets are distributed independently of each other. From he likelihood of observing n points $s_1, \dots, s_n$ within an area A is then


\[
\begin{array}{lll}
L(s_1, \dots, s_n) &= \exp\left(-\int_A \lambda(s) \, ds\right) \prod_{i=1}^n \lambda(s_i) & \text{for } n \geq 1 \\
L(n) &= \exp\left(-\int_A \lambda(s) \, ds\right) & \text{for } n = 0
\end{array}
\]



Furthermore, we can simulate the inhomogeneous Poisson process on an area A by taking the maximum intensity $\lambda_{max}$ on A, the simulating a homogeneous Poisson point process on A using this intensity, then for each simulated point $x_i$ deciding to accept it using the probability $\frac{\lambda(x_i)}{\lambda_{max}}$\cite{spatial_point_patterns}.



\end{comment}


%\section{Gaussian Random Fields}
%Gaussian random fields(GRFs) is a usefull tool in modelleing the interdependency between points in space. 

%Definition 2 : A random field $\{X(s):s \in S\}$ where $S\subset \mathbb{R}^n$ is said to be Gaussian, if for every finite subset $\Lambda \subset S$ and sequence $\{X(s):s\in \Lambda\}$, $\sum_{s\in \Lambda} X(s)$ is Gaussian\cite{gaetan_spatial_2010}.

%If a random field satisfies $E[X(s)^2]< \infty$ we call it a second order random field, and can define a covariance function $c(s,t) = Cov(X(s), X(t))$. from this we can construct the covariance matrix $\Sigma_{ij} = c(s_i,s_j)$ used in a GRF at postions $s_1, \dots s_k \in S$. The covariance function allows us to elegantly describe the large scale behaviour of the GRF. If the covariance function only depends on the distance between to points it is said to be isotropic. Throughout this thesis, the only covariance function used is the exponential covariance function which is isotropic and is give by

%$$
%    C(h;\sigma^2, \kappa, \phi) = \sigma^2 e^{-(h/\phi)^\kappa}
%$$

%\cite{gaetan_spatial_2010}. $\phi$ is known as the range parameter, $\kappa$ is the smoothness parameter, and $\sigma^2$ is the variance parameter. the varaiance parameter controls the magnitude of the covariance, the range parameter how fast the covariance declines with the increase in distance and the smoothness determines the rapidity of the decline. A high value for the smoothness parameter gives a field where there is little difference between close points, making the field appear more smooth.



\section{Monte Carlo Integration}
A common problem in statistics is that of taking an integral with respect to a stochastic variable. One such example is that of finding the expected value of a function of a variable $h$, $\mathbb{E}[h(X)]$. Often this is difficult, or impossible. In such cases the expectation has to be estimated instead. Let $f$ be the distribution $X$ is drawn from, the Monte Carlo method then finds the estimate 

\begin{equation}
\hat{\mu} = \frac{1}{N} \sum_{i=1}^N h(X_i)
\label{eq: monte carlo estimator}    
\end{equation}
  
for the expected value of h, $\mathbb{E}[h(X)]$. By the strong law of large numbers, this estimate converges to the actual expectation as $N\rightarrow\infty$. Let $v(x) = (h(x) - {\mu})^2$ and assume that $h(X)^2$ has a finite expectation under f, then we can estimate the sampling variance of \eqref{eq: monte carlo estimator} is $\sigma^2/N = \mathbb{E}[v(X)/N]$. $\sigma^2$ can be estimated by using the Monte Carlo estimator

\begin{equation}
    \widehat{Var}[\hat{\mu}] = \frac{1}{N-1}\sum_{i=1}^N (h(X_i) - \hat{\mu})^2
\end{equation}

If $\sigma^2$ exists, then the central limit theorem implies that $\hat{\mu}$ is approximately Gaussian distributed for large $N$. Because of this we can easily make confidence intervals for $\mu$.


\subsection{Importance Sampling}
\label{subsec: importance sampling theory}
Often, the variance of the estimator \eqref{eq: monte carlo estimator} can be large. This is especially true if events of interest for $h$ rarely occur under $f$. The idea of importance sampling is to draw from a different distribution $g$ which has higher probability for the events we are interested in, then adjusting for the fact that we are not sampling from $f$. The expected value $\mu$ can be written in an alternate form $\int h(x)f(x)dx = \int h(x)\frac{f(x)}{g(x)}g(x) dx$ where we are taking an expectation with respect to $g$, instead of $f$. The Monte Carlo estimator then becomes

\begin{equation}
    \hat{\mu} = \frac{1}{N} \sum_{i=1}^N h(X_i)\frac{f(X_i)}{g(X_i)}
\end{equation}

For this estimator to work, it is important that the support of $g$ be equal to the support of $f$
In addition if we want the estimator to reduce the variability, $\frac{f(x)}{g(x)}$ should be bounded, and $g$ should have heavier tails than $f$. Otherwise a draw from $g$ with low probability, but high probability for $f$ would give a large $\frac{f(x)}{g(x)}$ which would make that sample dominate the estimate. This in turn would give increased variance to the importance sampling estimator. Ideally then,  we want $\frac{f(x)}{g(x)}$ to be large only when $h(x)$ is small. \parencite{givens2013computational}






WRITE ABOUT WHAT CAN GO WRONG WITH THE IMPORTANCE SAMPLER
EFFECTIVE SAMPLE SIZE AND DEGENERACY
as time progresses in sequential importance sampling, the weights can degenerate
WRITE ABOUT HOW THE BIAS DECREASES WITH GREATER N

