\section{Introduction}

A common problem in ecology is finding the habitat preferences of a species. This is the tendency of an animal to prefer one area to another. mathematically, habitat preferences are described by the utilization distribution. this is defined as the probability of finding the animal in a particular area. various methods exist for estimating the utilization distribution. One such method is described in \cite{michelot_langevin_2019}. In this paper, the authors approximate a model of animal movement based on the Langevin diffusion equation, using animal tracking data. they then find the utilization distribution by finding the stationary distribution of the animal movement process. This method uses tracking data from animals to approximate the diffusion process. 

Covariates can be included in the utilization distribution by using a resource selection function. This allows for inference and interpretation to be made about the animals. In this way the habitat of the animals can be related to geographical 
\nocite{imt_software_wiki}  % This is an example of how to add a reference to the bibliography at the end without having it displayed as a reference within the text