

In conclusion, the bias seen in the Langevin movement model when observations are sparse can be removed by using Monte Carlo integration to integrate over intermediate states between observations. This can be done more efficiently by using importance sampling with Brownian motion with observations as endpoints as a proposal. However, this method requires selecting the number of bridges $M$ and the number of bridge nodes $N$ to be large enough for the method to converge. If $N$ is too low, the bridges will not accurately model the Langevin process. However, this can be compensated for by using a larger $M$. To verify whether the method gives an accurate estimate, the estimates have to be rerun at a higher value of $M$ or $N$. This can add additional computation time to the estimation process. For the Langevin movement model we have that a lower sampling frequency gives a lower variance for UD parameters. If the Langevin movement model is used to model animal movement, this fact can be used to improve experiments, since it is computationally feasible to obtain unbiased estimates of the UD parameters even when step intervals between samples are large.








 